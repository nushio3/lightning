\documentclass[pdftex]{beamer}
\AtBeginDvi{\special{pdf:tounicode EUC-UCS2}}

\usetheme{CambridgeUS} 
\usecolortheme{whale}

\usefonttheme{professionalfonts}   
\setbeamercovered{transparent}     

\setbeamertemplate{theorems}[numbered]  %% 定理に番号をつける
\newtheorem{thm}{Theorem}[section]
\newtheorem{proposition}[thm]{Proposition}
\theoremstyle{example}
\newtheorem{exam}[thm]{Example}
\newtheorem{remark}[thm]{Remark}
\newtheorem{question}[thm]{Question}
\newtheorem{prob}[thm]{Problem}

\begin{document}
\title[PPD Lightning]{Observation of Lightning in Protoplanetary Disks by Ion Lines} 
\author[Muranushi et. al.]{
Muranushi, T. ${}^1$
Akiyama, E. ${}^2$
Inutsuka, S. ${}^3$
Nomura, H. ${}^4$
and Okuzumi, S.${}^3$}         
\institute[Kyoto Univ.]{
${}^1$Kyoto University,
${}^2$The National Astronomical Observatory of Japan,
${}^3$Nagoya University,
${}^4$Tokyo Institute of Technology
}  
\date{May 1, 2014}



\begin{frame}             
\titlepage                
\end{frame}

\begin{frame}             
\tableofcontents
\end{frame}

% Insert document body here
% 
% \section{箇条書き}             %% セクション名
% \begin{frame}
% \frametitle{松本}              %% フレームタイトル
% 
% \begin{itemize}
% \item 豊科\pause               %% \pause でとまる
% \item 穂高\pause
% \item $E=mc^2$
% \end{itemize}
% \end{frame}
% 
% \section{定理型環境}           %% 定理型環境が使える
% \begin{frame}                  %% \newtheorem で新しい環境も作れる
% \begin{thm}
% 定理型環境が使える。
% 使い方は普通の \LaTeX と同じ
% \end{thm}
% \pause
% 
% \begin{proof}
% 証明も書ける。
% \end{proof}
% \pause
% 
% \begin{exam}                   %% 色が違う
% example
% \end{exam}
% \end{frame}
% 
% \section{文字の色}             %% 文字の色を変える
% \begin{frame}
% \frametitle{文字の色を変えてみよう}
% {\color{red}赤}\pause
% {\color{blue}青}\pause
% {\color{green}緑}
% \end{frame}
% 
\end{document}

